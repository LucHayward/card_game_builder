\documentclass[playing_cards, grid, fronts]{flashcards}
\cardbackstyle{empty}
\cardfrontstyle[\LARGE]{headings}
\usepackage{geometry}
\usepackage{xcolor}
\renewcommand*\abstractname{Instructions}

\begin{document}
    \begin{titlepage}
        \vspace*{2.5cm}
        \begin{minipage}{0.425\linewidth}
            \begin{flushright}
                \Huge{{game_name}}
            \end{flushright}
        \end{minipage}
        % Leave some space between the left minipage and the vertical bar
        \hspace{15pt}
        % Minipage for the vertical bar
        \begin{minipage}{0.02\linewidth}
            \rule{1pt}{75pt}
        \end{minipage}
        \hspace{5pt}
        % begin right minipage
        \begin{minipage}{0.40\linewidth}
            \begin{flushleft}

                \normalsize
                Number of players: {num_players}

                Duration: {duration}

                Authors: {authors}

                Website: {website}
            \end{flushleft}
        \end{minipage}

        \vspace*{1.5cm}
        \par
        % begin left minipage
        % Minipage for the instructions
        \centering{
            \begin{minipage}{0.5\linewidth}
                {instructions}
            \end{minipage}
        }
        % Add a footer to indicate where the card game came from
        \par
        \vspace*{\fill}
        \centering{
            \begin{minipage}{\linewidth}
                \centering
                This PDF was created by Boyd Kane's Card Game Builder on {date}.

                You can use the Card Game Builder to create your own card

                games (or to print out existing card games) by going to

                \texttt{www.github.com/beyarkay/card\_game\_builder}
            \end{minipage}
        }
        \vspace*{\fill}
    \end{titlepage}
    \cardfrontfoot{\color{lightgray}{{expansion_name}}}
{flashcard_items}
\end{document}
